\def\duedate{09/29/22}
\def\HWnum{1}
% Document setup
\documentclass[12pt]{article}
\usepackage[margin=1in]{geometry}
\usepackage{fancyhdr}
\usepackage{lastpage}

\pagestyle{fancy}
\lhead{Richard Whitehill}
\chead{MATH 757 -- HW \HWnum}
\rhead{\duedate}
\cfoot{\thepage \hspace{1pt} of \pageref{LastPage}}

% Encoding
\usepackage[utf8]{inputenc}
\usepackage[T1]{fontenc}

% Math/Physics Packages
\usepackage{amsmath}
\usepackage{amssymb}
\usepackage{dsfont}
\usepackage{mathtools}
\usepackage[arrowdel]{physics}
\usepackage{siunitx}

\AtBeginDocument{\RenewCommandCopy\qty\SI}

% Reference Style
\usepackage{hyperref}
\hypersetup{
    colorlinks=true,
    linkcolor=blue,
    filecolor=magenta,
    urlcolor=cyan,
    citecolor=green
}

\newcommand{\eref}[1]{Eq.~(\ref{eq:#1})}
\newcommand{\erefs}[2]{Eqs.~(\ref{eq:#1})--(\ref{eq:#2})}

\newcommand{\fref}[1]{Fig.~\ref{fig:#1}}
\newcommand{\frefs}[2]{Figs.~\ref{fig:#1}--\ref{fig:#2}}

\newcommand{\tref}[1]{Table~\ref{tab:#1}}
\newcommand{\trefs}[2]{Tables~\ref{tab:#1}-\ref{tab:#2}}

% Figures and Tables 
\usepackage{graphicx}
\usepackage{float}

\newcommand{\bef}{\begin{figure}[h!]\begin{center}}
\newcommand{\eef}{\end{center}\end{figure}}

\newcommand{\bet}{\begin{table}[h!]\begin{center}}
\newcommand{\eet}{\end{center}\end{table}}

% tikz
\usepackage{tikz}
\usetikzlibrary{calc}
\usetikzlibrary{decorations.pathmorphing}
\usetikzlibrary{decorations.markings}
\usetikzlibrary{arrows.meta}
\usetikzlibrary{positioning}

% tcolorbox
\usepackage[most]{tcolorbox}
\usepackage{xcolor}
\usepackage{xifthen}
\usepackage{parskip}

\newcommand*{\eqbox}{\tcboxmath[
    enhanced,
    colback=black!10!white,
    colframe=black,
    sharp corners,
    size=fbox,
    boxsep=8pt,
    boxrule=1pt
]}

% code box
\usepackage{listings}

\definecolor{background}{RGB}{245,245,246}
\definecolor{rule}{RGB}{224,224,224}
\definecolor{keyword}{RGB}{0,128,0}
\definecolor{comment}{RGB}{172,191,206}
\definecolor{CadetBlue}{RGB}{97,110,196}
\definecolor{Red}{RGB}{255,0,0}

\lstdefinestyle{mystyle}{
    frame=trbl,
    basicstyle=\footnotesize,
    backgroundcolor=\color{background},
    rulecolor=\color{rule},
    commentstyle=\color{CadetBlue},
    keywordstyle=\color{keyword}\bfseries,
    keywordstyle={[2]\color{keyword}},
    numberstyle=\color{Green},
    stringstyle=\color{Red},
    commentstyle=\color{CadetBlue},
    showstringspaces=false,
    breakatwhitespace=false,
    breaklines=true,
    captionpos=b,
    keepspaces=true,
    numbersep=5pt,
    showspaces=false,
    showtabs=false,
    tabsize=2,
    columns=fixed,
}

\lstset{style=mystyle}

\lstnewenvironment{python}{
\lstset{
	language=Python,
	otherkeywords={as},
	emph={self},
	emphstyle=\color{Blue}
}
}
{}

\NewDocumentCommand{\inputpython}{m}{\lstinputlisting[language=Python]{#1}}

% Miscellaneous Definitions/Settings
\newcommand{\prob}[2]{\textbf{#1)} #2}

\setlength{\parskip}{\baselineskip}
\setlength{\parindent}{0pt}

\def\complexs{\mathbb{C}}
\def\reals{\mathbb{R}}
\def\naturals{\mathbb{N}}
\def\integers{\mathbb{Z}}
\def\rationals{\mathbb{Q}}
\def\id{\mathds{1}}



\setlength{\headheight}{14.9998pt}
\addtolength{\topmargin}{-2.49998pt}

\begin{document}
    
\prob{1}{
Assume $x$ is an $m$-dimensional vector, prove $||x||_{\infty} \leq ||x||_{2}$ and $||x||_{2} \leq \sqrt{m} ||x||_{\infty}$.
}

Recall that 
\begin{align}
    \label{eq:norm-def}
    ||x||_{2} &= \sqrt{\sum_{i=1}^{m} |x_{i}|^{2}} \\
    ||x||_{\infty} &= \max \{ |x_1|,|x_2|, \ldots, |x_{m}| \} 
.\end{align}

Thus, for the first inequality we have the following.
Suppose that $|x_{k}| = ||x||_{\infty}$, then
\begin{eqnarray}
    \label{eq:derive-1st-inequality}
    \eqbox{
    ||x||_{2} = \sqrt{||x||_{\infty}^2 + \sum_{i \ne k} |x_{i}|^2} \leq \sqrt{||x||_{\infty}^2} \leq ||x||_{\infty}
}
,\end{eqnarray}
since $|x_{i}| \geq 0$ for all $i$.

For the second inequality, we know that $|x_{i}| \leq ||x||_{\infty}$ for all $i$.
Hence,
\begin{eqnarray}
    \label{eq:derive-2nd-inequality}
    \eqbox{
    ||x||_{2} \leq \sqrt{\sum_{i=1}^{m} ||x||_{\infty}^2} = \sqrt{m}||x||_{\infty} 
}
.\end{eqnarray}



\prob{2}{
Show that if a matrix $A$ is both triangular and unitary, then it is diagonal.
}

Suppose that $A = (a_{ij}) \in \complexs^{n \times n}$ is an upper triangular matrix we know that $a_{ij} = 0$ if $i > j$.
Furthermore, we know that $A A^{\dagger} = A^{\dagger}A = 1$.
Using the definition of matrix multiplication, we see that
\begin{eqnarray}
    \label{eq:A-Adag}
    (A^{\dagger} A)_{ij} = \sum_{k=1}^{n} a_{ki}^{*}a_{kj} = \delta_{ij} 
,\end{eqnarray}
where $\displaystyle \delta_{ij} = \begin{cases}
    1 & i = j \\
    0 & i \ne j
\end{cases}$.
Since $A$ is upper triangular, it immediately follows that $A^{\dagger}$ is lower triangular. 
Hence, we find that $(A^{\dagger}A)_{11} = 1 = |a_{11}|^2$ or that $|a_{11}| = 1$.
Next, we observe that $(A^{\dagger} A)_{12} = 0 = a_{11}^{*}a_{12}$, which implies that $a_{12} = 0$.
Suppose that it is true that $a_{12},\ldots,a_{1 m} = 0$ for some $m < n$, then $(A^{\dagger} A)_{1,m+1} = 0 = a_{1 1}^{*}a_{1,m + 1}$, which implies that $a_{1,m+1} = 0$.
This argument can be continued for the elements above the diagonal of $A$ for each row.
Hence, $a_{ij} = 0$ if $i > j$.

Now suppose that $A$ is lower triangular and unitary.
Then, $A^{\dagger}$ is upper triangular and unitary, which implies that $A^{\dagger}$ is diagonal.
It immediately follows then that $A$ is diagonal. 

\prob{3}{
Prove that matrix $\infty$-norm is
\begin{eqnarray}
    \label{eq:infty-norm-matrix}
    ||A||_{\infty} = \max_{i} \sum_{j=1}^{n} |a_{ij}|
.\end{eqnarray}
}



\prob{4}{
    Let $|| \cdot ||$ denote any norm on $\complexs$ and also the induced matrix norm on $\complexs^{m \times m}$.
Show that $\rho(A) \leq ||A||$, where $\rho$ is the spectral radius of $A$, which is the largest absolute value of an eigenvalue of $A$.
}


\prob{5}{
Find $l_{1}$, $l_{2}$, and $l_{\infty}$ norms of the following vectors and matrices, also, please verify your results by MATLAB.
}

(a) $x = (3,-4,0,\frac{3}{2})^{\rm T}$

(b) $\begin{pmatrix}
    2 & -1 & 0 \\
    -1 & 2 & -1 \\
    0 & -1 & 2
\end{pmatrix}
$


\prob{6}{
Determine SVDs of the following matrices by hand calculation and MATLAB.
}

(a) $\begin{pmatrix}
    3 & 0 \\
    0 & -2
\end{pmatrix}
$
(b) $
\begin{pmatrix}
    0 & 2 \\
    0 & 0 \\
    0 & 0 \\
    0 & 0
\end{pmatrix}
$


\end{document}
