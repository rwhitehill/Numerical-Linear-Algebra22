\def\duedate{09/29/22}
\def\HWnum{1}
% Document setup
\documentclass[12pt]{article}
\usepackage[margin=1in]{geometry}
\usepackage{fancyhdr}
\usepackage{lastpage}

\pagestyle{fancy}
\lhead{Richard Whitehill}
\chead{PHYS 714 -- HW \HWnum}
\rhead{\duedate}
\cfoot{\thepage \hspace{1pt} of \pageref{LastPage}}

% Encoding
\usepackage[utf8]{inputenc}
\usepackage[T1]{fontenc}

% Math/Physics Packages
\usepackage{amsmath}
\usepackage{amssymb}
\usepackage{dsfont}
\usepackage{mathtools}
\usepackage[arrowdel]{physics}
\usepackage{siunitx}

\AtBeginDocument{\RenewCommandCopy\qty\SI}

% Reference Style
\usepackage{hyperref}
\hypersetup{
    colorlinks=true,
    linkcolor=blue,
    filecolor=magenta,
    urlcolor=cyan,
    citecolor=green
}

\newcommand{\eref}[1]{Eq.~(\ref{eq:#1})}
\newcommand{\erefs}[2]{Eqs.~(\ref{eq:#1})--(\ref{eq:#2})}

\newcommand{\fref}[1]{Fig.~\ref{fig:#1}}
\newcommand{\frefs}[2]{Figs.~\ref{fig:#1}--\ref{fig:#2}}

\newcommand{\tref}[1]{Table~\ref{tab:#1}}
\newcommand{\trefs}[2]{Tables~\ref{tab:#1}-\ref{tab:#2}}

% Figures and Tables 
\usepackage{graphicx}
\usepackage{float}

\newcommand{\bef}{\begin{figure}[h!]\begin{center}}
\newcommand{\eef}{\end{center}\end{figure}}

\newcommand{\bet}{\begin{table}[h!]\begin{center}}
\newcommand{\eet}{\end{center}\end{table}}

% tikz
\usepackage{tikz}
\usetikzlibrary{calc}
\usetikzlibrary{decorations.pathmorphing}
\usetikzlibrary{decorations.markings}
\usetikzlibrary{arrows.meta}
\usetikzlibrary{positioning}

% tcolorbox
\usepackage[most]{tcolorbox}
\usepackage{xcolor}
\usepackage{xifthen}
\usepackage{parskip}

\newcommand*{\eqbox}{\tcboxmath[
    enhanced,
    colback=black!10!white,
    colframe=black,
    sharp corners,
    size=fbox,
    boxsep=8pt,
    boxrule=1pt
]}

% Miscellaneous Definitions/Settings
\newcommand{\prob}[2]{\textbf{#1)} #2}

\setlength{\parskip}{\baselineskip}
\setlength{\parindent}{0pt}

\def\complexs{\mathbb{C}}
\def\reals{\mathbb{R}}
\def\naturals{\mathbb{N}}
\def\integers{\mathbb{Z}}
\def\rationals{\mathbb{Q}}
\def\id{\mathds{1}}



\setlength{\headheight}{14.9998pt}
\addtolength{\topmargin}{-2.49998pt}

\begin{document}
    
\prob{1}{
Assume $x$ is an $m$-dimensional vector, prove $||x||_{\infty} \leq ||x||_{2}$ and $||x||_{2} \leq \sqrt{m} ||x||_{\infty}$.
}

Recall that 
\begin{align}
    \label{eq:norm-def}
    ||x||_{2} &= \sqrt{\sum_{i=1}^{m} |x_{i}|^{2}} \\
    ||x||_{\infty} &= \max \{ |x_1|,|x_2|, \ldots, |x_{m}| \} 
.\end{align}

Thus, for the first inequality we have the following.
Suppose that $|x_{k}| = ||x||_{\infty}$, then
\begin{eqnarray}
    \label{eq:derive-1st-inequality}
    \eqbox{
    ||x||_{2} = \sqrt{||x||_{\infty}^2 + \sum_{i \ne k} |x_{i}|^2} \leq \sqrt{||x||_{\infty}^2} \leq ||x||_{\infty}
}
,\end{eqnarray}
since $|x_{i}| \geq 0$ for all $i$.

For the second inequality, we know that $|x_{i}| \leq ||x||_{\infty}$ for all $i$.
Hence,
\begin{eqnarray}
    \label{eq:derive-2nd-inequality}
    \eqbox{
    ||x||_{2} \leq \sqrt{\sum_{i=1}^{m} ||x||_{\infty}^2} = \sqrt{m}||x||_{\infty} 
}
.\end{eqnarray}



\prob{2}{
Show that if a matrix $A$ is both triangular and unitary, then it is diagonal.
}

Suppose that $A = (a_{ij}) \in \complexs^{n \times n}$ is an upper triangular matrix we know that $a_{ij} = 0$ if $i > j$.
First, we will show that $A^{-1}$, assuming that $A$ is non-singular, is upper triangular.
Let $A^{-1} = [b_1 b_2 \ldots b_{n}]$, then $A^{-1}A = \id$ and
\begin{eqnarray}
    \label{eq:inverse-equations}
    \begin{cases}
    b_1a_{11} = e_1 \\
    b_2a_{12} + b_2a_{22} = e_2 \\
    \vdots \\
    b_1a_{1n} + b_2a_{2n} + \ldots + b_{n}a_{nn} = e_{n}
    ,\end{cases} 
\end{eqnarray}
where $(e_{i})_{j} = \delta_{ij} = \begin{cases}
    1 & i = j \\
    0 & i \ne j
\end{cases}$
is the standard basis.
From lectures, we know that we can solve this system using forward substitution as follows:
\begin{eqnarray}
    \label{eq:backsub}
    \begin{cases}
    b_1 = e_1/a_{11} \\
    b_{i} = (e_{i} - \sum_{k=1}^{i-1} b_{k}a_{ki})/a_{ii}
    .\end{cases} 
\end{eqnarray}
Observe that
\begin{align}
    \label{eq:prove-upper-triangle}
    (b_1)_{j} &= \frac{1}{a_{11}}\delta_{1j} \\
    (b_{i})_{j} &= \frac{1}{a_{ii}}\delta_{ij} - \sum_{k=1}^{i-1} (b_{k})_{j}\frac{a_{ki}}{a_{ii}}
.\end{align}
Using induction, it is easy to see that $b_{ij} = 0$ if $i > j$.

Next, we will show that $A^{\rm T}$ must be lower triangular, which implies that $A^{\dagger} = A^{* {\rm T}}$ is lower triangular.
From the definition $(A^{\rm T})_{ij} = a_{ji}$.
Since $a_{ji} = 0$ if $j > i$, then it follows that $(A^{\rm T})_{ij} = 0$ if $j > i$, which proves that $A^{\rm T}$ is lower triangular.

Now, let $A$ be a unitary.
Then, $A^{\dagger} = A^{-1}$.
Hence,
\begin{eqnarray}
    \label{eq:Adag-Ainv}
    (A^{\dagger})_{ij} = a^{*}_{ji} = (A^{-1})_{ij}
.\end{eqnarray}
Recall that the inverse of $A$ is upper triangular, so
\begin{eqnarray}
    \label{eq:prove-diagonal}
    a^{*}_{ji} = 0
\end{eqnarray}
if $i > j$.
That is, $a_{ji} = 0$ if $j < i$, meaning that $A$ has no off-diagonal entries and that $A$ is diagonal.
Furthermore, it is clear now that $|a_{ii}|^{2} = 1$ or $|a_{ii}| = 1$, which gives a constraint for the diagonal values of $A$.

The proof for a lower triangular matrix is simple. 
Let $A$ be a lower triangular unitary matrix.
Then $A^{\dagger}$ is upper triangular and unitary since $A^{\dagger} A = \id$, which we showed above must imply that $A^{\dagger}$ is diagonal.
Hence, $A$ must be diagonal as well.


\prob{3}{
Prove that matrix $\infty$-norm is
\begin{eqnarray}
    \label{eq:infty-norm-matrix}
    ||A||_{\infty} = \max_{i} \sum_{j=1}^{n} |a_{ij}|
.\end{eqnarray}
}



\prob{4}{
    Let $|| \cdot ||$ denote any norm on $\complexs$ and also the induced matrix norm on $\complexs^{m \times m}$.
Show that $\rho(A) \leq ||A||$, where $\rho$ is the spectral radius of $A$, which is the largest absolute value of an eigenvalue of $A$.
}


\prob{5}{
Find $l_{1}$, $l_{2}$, and $l_{\infty}$ norms of the following vectors and matrices, also, please verify your results by MATLAB.
}

(a) $x = (3,-4,0,\frac{3}{2})^{\rm T}$

(b) $\begin{pmatrix}
    2 & -1 & 0 \\
    -1 & 2 & -1 \\
    0 & -1 & 2
\end{pmatrix}
$


\prob{6}{
Determine SVDs of the following matrices by hand calculation and MATLAB.
}

(a) $\begin{pmatrix}
    3 & 0 \\
    0 & -2
\end{pmatrix}
$
(b) $
\begin{pmatrix}
    0 & 2 \\
    0 & 0 \\
    0 & 0 \\
    0 & 0
\end{pmatrix}
$


\end{document}
